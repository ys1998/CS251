\documentclass[titlepage,11pt,a4paper]{article}
\usepackage{listings}
\usepackage[hidelinks]{hyperref}
\usepackage{lmodern}
\begin{document}

\hypersetup{
    colorlinks=false
}
\urlstyle{same}

\title{\bf \Huge CS251 \LaTeX}
\date{10 August, 2017}
\author{\LARGE Contour}
\maketitle

\tableofcontents


\newpage


\section{About Me}
Hello, My name is Diptesh Kanojia. I am currently pursuing PhD in Computational
Linguistics. This is a \LaTeX document for the course \textbf{Software Systems Lab}
with course $cs251$. I would like for this document to be typesetted perfectly which
forces me to you use \LaTeX. \LaTeX uses various packages. I will elaborate about
them in the following subsections:
\subsection{graphicx package}
This package is used to import tables, and figure in the document. Our document type
is article, and we are currently using 11pt font size, with
a4 type paper, which is specified in the beginning in $<$documentclass$>$ .

\subsection{amssymb package}
This package is used to import mathematical symbols in the document. We encapsulate
the mathematical equations and symbols under \$, and they are changed to maths
symbols.
\section{Some History}
I am ancient creature dwelling on this planet now referred to as ’Earth’. I have
been existing since the past $150393894.5$ years. Do you see the use of a package
above in the number mention in the document. I have used something to enunciate the
numbers in a fashion such as a mathematical formulae.

\indent Let us all try to replicate the text provided in this document.

\indent \textit{P.S.:Please note that I am following the Section Title \textbf{Noun
Capitalization} in the document. This would be followed in the rest of the document,
henceforth.}

\section{Replication}
\paragraph{}
\textbf{LaTeX} is a word processor and document markup language. It is distinguished from typical word processors such as Microsoft Word and Apple Pages in that the writer uses plain text as opposed to formatted text, relying on markup tagging conventions to define the general structure of a document (such as article, book, and letter), to stylise text throughout a document (such as \textbf{bold} and \textit{italic}), and to add citations and cross-referencing. A {\bf TeX} distribution such as {\bf TeXlive} or {\bf MikTeX} is used to produce an output file (such as PDF or DVI) suitable for printing or digital distribution.

\paragraph{}
\textbf{LaTeX} is used for the communication and publication of scientific documents in many fields, including mathematics, physics, computer science, statistics, economics, and political science. It also has a prominent role in the preparation and publication of books and articles that contain complex multilingual materials, such as Sanskrit and Arabic. \textbf{LaTeX} uses the TeX typesetting program for formatting its output, and is itself written in the TeX macro language.

\paragraph{}
\textbf{LaTeX} is widely used in academia. \textbf{LaTeX} can be used as a standalone document preparation system, or as an intermediate format. In the latter role, for example, it is often used as part of a pipeline for translating DocBook and other XML-based formats to PDF. The type-setting system offers programmable desktop publishing features and extensive facilities for automating most aspects of typesetting and desk- top publishing, including numbering and cross-referencing of tables and figures, chapter and section headings, the inclusion of graphics, page layout, indexing and bibliographies.

\paragraph{}
Like {\bf TeX}, \textbf{LaTeX} started as a writing tool for mathematicians and computer scientists, but from early in its development it has also been taken up by scholars who needed to write documents that include complex math expressions or non-Latin scripts, such as Arabic, Sanskrit%
    and Chinese.

\paragraph{}
\textbf{LaTeX} is intended to provide a high-level language that accesses the power of {\bf TeX}. \LaTeX comprises a collection of TeX macros and a program to process \textbf{LaTeX} documents. Because the plain {\bf TeX} formatting commands are elementary, it provides authors with ready-made commands for formatting and layout requirements such as chapter headings, foot-notes, cross-references and bibliographies.

\paragraph{}
\textbf{LaTeX} was originally written in the early 1980s by Leslie Lamport at SRI International. The current version is \textbf{LaTeX}2e. \textbf{LaTeX} is free software and is distributed under the \textbf{LaTeX} Project Public License (LPPL) \textbf{(Source \href{https://en.wikipedia.org/wiki/LaTeX}{Wikipedia})}.
 
\section{Opening and Compiling Tex Document}
First create a {\bf .tex} file using text editor such as {\bf Vi} or {\bf Gedit} or {\bf Kile}.
    
\section{Starting and Ending}
A minimal input file looks like following 
\begin{lstlisting}[moredelim={[is][keywordstyle]{@@}{@@}}, basicstyle=\ttfamily]
     @@\documentclass{class}
     \begin{document}
     your text...
     \end {document}@@
\end{lstlisting}
where the class is a valid document class for \textbf{LaTeX}.

\subsection{Compiling the LaTeX Document}
We open the terminal and go to the directory in which our .tex file is stored and the we execute the command

\begin{lstlisting}[escapechar=@, basicstyle=\ttfamily]
     @\textbf{pdflatex example.tex}@
\end{lstlisting}

\section{Section} \label{s6}
Sectioning commands provide the means to structure your text into units:

% \begin{lstlisting}[moredelim={[is][keywordstyle]{@@}{@@}}, basicstyle=\ttfamily]
% @@\part
% \chapter
% (report and book class only)
% \section
% \subsection
% \subsubsection
% \paragraph
% \subparagraph@@
% \end{lstlisting}
\paragraph{\textbackslash part}
\paragraph{\textbackslash chapter}
\paragraph{(report and book class only)}
\paragraph{\textbackslash section}
\paragraph{\textbackslash subsection}
\paragraph{\textbackslash subsubsection}
\paragraph{\textbackslash paragraph}
\paragraph{\textbackslash subparagraph}\ \\

\indent All sectioning commands take the same general form, e.g.,
\begin{lstlisting}[moredelim={[is][keywordstyle]{@@}{@@}}, basicstyle=\ttfamily]
    @@\chapter[toctitle]{title}@@
\end{lstlisting}

\indent In addition to providing the heading title in the main text, the section
title can appear in two other places:

\begin{enumerate}
\item The table of contents.
\item The running head at the top of the page.
\end{enumerate}

\paragraph{}
You may not want the same text in these places as in the main text.
To handle this, the sectioning commands have an optional argument
toctitle that, when given, specifies the text for these other places.

\paragraph{}
Also, all sectioning commands have *-forms that print title as usual,
butdo not include a number and do not make an entry in the table of
contents.

\paragraph{}
For instance:
\begin{lstlisting}[moredelim={[is][keywordstyle]{@@}{@@}}, basicstyle=\ttfamily]
	@@\section*{Preamble}@@
\end{lstlisting}

\paragraph{}
The \textbf{\textbackslash appendix} command changes the way following sectional units
are numbered. The \textbf{\textbackslash appendix} command itself generates no text and
does not affect the numbering of parts.

\paragraph{}
The normal use of this command is something like

\begin{lstlisting}[moredelim={[is][keywordstyle]{@@}{@@}}, basicstyle=\ttfamily]
	@@\chapter{A Chapter}
	...
	\appendix
	\chapter{The First Appendix}@@
\end{lstlisting}

\indent The secnumdepth counter controls printing of section numbers. The
setting suppresses heading numbers at any depth $>$ level, where chap-
ter is level zero.
\begin{lstlisting}[moredelim={[is][keywordstyle]{@@}{@@}}, basicstyle=\ttfamily]
	@@\setcounter{secnumdepth}{level}@@
\end{lstlisting}

\section{Cross Reference}
One reason for numbering things like figures and equations is to refer
the reader to them, as in ``Section 6 on Page \ref{s6} for more details". {\Large I have referred to a section, and a page here here.}

\subsection{\textbackslash label\{key\}}
A \textbf{\textbackslash label} command appearing in ordinary text assigns to key the number of the current sectional unit; one appearing inside a numbered environment assigns that number to key. 
\paragraph{} A key name can consist of any sequence of letters, digits, or punctuation characters. Upper and lowercase letters are distinguished.
\paragraph{} To avoid accidentally creating two labels with the same name, it is common to use labels consisting of a prefix and a suffix separated by a colon or period. Some conventionally-used prefixes:\\
\textbf{ch}\hspace{1cm}for chapters\\
\textbf{sec}\hspace{1cm}for lower-level sectioning commands\\
\textbf{fig}\hspace{1cm}for figures\\
\textbf{tab}\hspace{1cm}for tables\\
\textbf{eq}\hspace{1cm}for equations\\

\paragraph{}{\Huge I think we have replicated the document enough. Let us just concentrate on learning features of the document provided to us. We have successfully demonstrated the the features such as Sections, Subsections, Labelling, Bold, Italics, Tabbing, Title Page, Huge, Large, mathsymbols. Typing in a \LaTeX document to type in \textbf{LaTeX} code.}
\paragraph{} Let us leave the rest of \LaTeX for the out-lab.
\end{document}